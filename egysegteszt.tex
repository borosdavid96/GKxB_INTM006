\documentclass[usenames,dvipsnames,aspectratio=169]{beamer}

\usepackage[utf8]{inputenc}
\usepackage[T1]{fontenc}
\usepackage[magyar]{babel}
\usepackage{indentfirst}
\usepackage{listingsutf8}
\lstset{literate=
  {á}{{\'a}}1 {é}{{\'e}}1 {í}{{\'i}}1 {ó}{{\'o}}1 {ú}{{\'u}}1
  {Á}{{\'A}}1 {É}{{\'E}}1 {Í}{{\'I}}1 {Ó}{{\'O}}1 {Ú}{{\'U}}1
  {à}{{\`a}}1 {è}{{\`e}}1 {ì}{{\`i}}1 {ò}{{\`o}}1 {ù}{{\`u}}1
  {À}{{\`A}}1 {È}{{\'E}}1 {Ì}{{\`I}}1 {Ò}{{\`O}}1 {Ù}{{\`U}}1
  {ä}{{\"a}}1 {ë}{{\"e}}1 {ï}{{\"i}}1 {ö}{{\"o}}1 {ü}{{\"u}}1
  {Ä}{{\"A}}1 {Ë}{{\"E}}1 {Ï}{{\"I}}1 {Ö}{{\"O}}1 {Ü}{{\"U}}1
  {â}{{\^a}}1 {ê}{{\^e}}1 {î}{{\^i}}1 {ô}{{\^o}}1 {û}{{\^u}}1
  {Â}{{\^A}}1 {Ê}{{\^E}}1 {Î}{{\^I}}1 {Ô}{{\^O}}1 {Û}{{\^U}}1
  {œ}{{\oe}}1 {Œ}{{\OE}}1 {æ}{{\ae}}1 {Æ}{{\AE}}1 {ß}{{\ss}}1
  {ç}{{\c c}}1 {Ç}{{\c C}}1 {ø}{{\o}}1 {å}{{\r a}}1 {Å}{{\r A}}1
  {€}{{\EUR}}1 {£}{{\pounds}}1 {ő}{{\H{o}}}1
}
\lstdefinestyle{cpp}{
  language=[ISO]C++,
  showstringspaces=false,
  keywordstyle=\color{MidnightBlue}\bfseries,
  stringstyle=\color{DarkOrchid},
  commentstyle=\color{Brown},
  morecomment=[l][\color{OliveGreen}]{\#}
}
\usepackage{hyperref}
\usepackage{attachfile}
\usepackage{multirow}
\attachfilesetup{color={1.0 0.6 0.0},author={HFM},description={Kattintson duplán a minta %
megtekintéséhez!},icon=Paperclip}
\usetheme{Warsaw}
\definecolor{kiemelesszin}{rgb}{0.6,0.0,0.0}
\definecolor{kiemelesszinZ}{rgb}{0.0,0.6,0.0}
\definecolor{kiemelesszinN}{RGB}{196,127,0}
\definecolor{hivatkozasszin}{rgb}{0.0,0.0,0.75}
\newcommand{\kiemel}[1]{{\color{kiemelesszin}#1}}
\newcommand{\kiemelZ}[1]{{\color{kiemelesszinZ}#1}}
\newcommand{\kiemelN}[1]{{\color{kiemelesszinN}#1}}
\newcommand{\hiv}[1]{{\color{hivatkozasszin}#1}}
\frenchspacing
\usetheme{Berlin}

\title[Modern szoftverfejlesztési eszközök - egységtesztek]{C++ programok egységtesztelése googletest segítségével}
\subtitle{(GKxB\_INTM006)}
\author{Dr. Hatwágner F. Miklós}
\institute{Széchenyi István Egyetem, Győr}

\begin{document}

%1
\begin{frame}[plain]
  \titlepage
\end{frame}

\section{Dióhéjban a tesztelésről}

\begin{frame}
  Tesztelés célja: a hibákat megtalálni üzembe helyezés előtt
  \vfill
  Tesztelés alapelvei
  \begin{enumerate}
    \item A tesztelés bizonyos hibák jelenlétét jelezheti (ha nem jelzi, az nem jelent automatikusan hibamentességet)
    \item Nem lehetséges kimerítő teszt (a hangsúly a magas kockázatú részeken van)
    \item Korai teszt (minél hamarabb találjuk meg a hibát, annál olcsóbb javítani)
    \item Hibák csoportosulása (azokra a modulokra/bemenetekre kell tesztelni, amelyre a legvalószínűbben hibás a szoftver)
    \item Féregirtó paradoxon (a tesztesetek halmazát időnként bővíteni kell, mert ugyanazokkal a tesztekkel nem fedhetünk fel 
több hibát)
    \item Körülmények (tesztelés alapossága függ a felhasználás helyétől, a rendelkezésre álló időtől, stb.)
    \item A hibátlan rendszer téveszméje (A megrendelő elsősorban az igényeinek megfelelő szoftvert szeretne, és csak 
másodsorban hibamenteset; verifikáció vs. validáció)
  \end{enumerate}
\end{frame}

\begin{frame}
  Tesztelési technikák
  \begin{description}[mm]
    \item[Fekete dobozos (black-box, specifikáció alapú)] \hfill\\ A tesztelő nem látja a forrást, de a specifikációt igen, és 
hozzáfér a futtatható szoftverhez. Összehasonlítjuk a bemenetekre adott kimeneteket az elvárt kimenetekkel.
    \item[Fehér dobozos (white-box, strukturális teszt)] \hfill\\ Kész struktúrákat tesztelünk, pl.:
    \begin{itemize}
      \item kódsorok,
      \item elágazások,
      \item metódusok,
      \item osztályok,
      \item funkciók,
      \item modulok.
    \end{itemize}
    Lefedettség: a struktúra hány \%-át tudjuk tesztelni a tesztesetekkel?\\
    Egységteszt (unit test): a metódusok struktúra tesztje.
  \end{description}
\end{frame}

\begin{frame}
  A tesztelés szintjei:
  \begin{enumerate}
    \item komponensteszt (egy komponens tesztelése)
    \begin{enumerate}
      \item egységteszt
      \item modulteszt
    \end{enumerate}
    \item integrációs teszt (kettő vagy több komponens együttműködése)
    \item rendszerteszt (minden komponens együtt)
    \item átvételi teszt (kész rendszer)
  \end{enumerate}
\end{frame}

\begin{frame}
  Kik végzik a tesztelést?
  \begin{itemize}
    \item[1-3] Fejlesztő cég
    \item[4] Felhasználók
  \end{itemize}
  Komponensteszt
  \begin{itemize}
    \item fehér dobozos teszt
    \item egységteszt
    \begin{itemize}
      \item bemenet $\to$ kimenet vizsgálata
      \item nem lehet mellékhatása
      \item regressziós teszt: módosítással elronthattunk valamit, ami eddig jó volt $\to$ megismételt egységtesztek
    \end{itemize}
    \item modulteszt
    \begin{itemize}
      \item nem funkcionális tulajdonságok: sebesség, memóriaszivárgás (memory leak), szűk keresztmetszetek (bottleneck)
    \end{itemize}
  \end{itemize}
\end{frame}

\begin{frame}
  Integrációs teszt
  \begin{itemize}
    \item Komponensek közötti interfészek ellenőrzése, pl.
    \begin{itemize}
      \item komponens - komponens (egy rendszer komponenseinek együttműködése)
      \item rendszer - rendszer (pl. OS és a fejlesztett rendszer között)
    \end{itemize}
    \item Jellemző hibaokok: komponenseket eltérő csapatok fejlesztik, elégtelen kommunikáció
    \item Kockázatok csökkentése: mielőbbi integrációs tesztekkel
  \end{itemize}
\end{frame}

\begin{frame}
  Rendszerteszt: a termék megfelel-e a
  \begin{itemize}
    \item követelmény specifikációnak,
    \item funkcionális specifikációnak,
    \item rendszertervnek.
  \end{itemize}
  Gyakran fekete dobozos, külső cég végzi (elfogulatlanság)\\
  Leendő futtatási környezet imitációja
\end{frame}

\begin{frame}
  Átvételi teszt, fajtái:
  \begin{itemize}
    \item alfa: kész termék tesztelése a fejlesztőnél, de nem általa (pl. segédprogramok)
    \item béta: szűk végfelhasználói csoport
    \item felhasználói átvételi teszt: minden felhasználó használja, de nem éles termelésben. Jellemző a környezetfüggő hibák 
megjelenése (pl. sebesség)
    \item üzemeltetői átvételi teszt: rendszergazdák végzik, biztonsági mentés, helyreállítás, stb. helyesen működnek-e
  \end{itemize}
\end{frame}

\section{C++ egységtesztelés}

\begin{frame}
  Rengeteg C++ egységteszt keretrendszerből lehet választani:
  \begin{itemize}
    \item \hiv{\href{https://en.wikipedia.org/wiki/List\_of\_unit\_testing\_frameworks\#C++}{Wiki oldal}}
    \item \hiv{\href{http://gamesfromwithin.com/exploring-the-c-unit-testing-framework-jungle}%
      {Exploring the C++ Unit Testing Framework Jungle}}
    \item \hiv{\href{https://accu.org/index.php/journals/1326}{C++ Unit Test Frameworks}}
  \end{itemize}
  \vfill
  Részletesen megvizsgáljuk: googletest
\end{frame}

\section{googletest}

\begin{frame}
  A googletest főbb tulajdonságai
  \begin{itemize}
    \item platformfüggetlen (Linux, Windows, Mac)
    \item független és megismételhető tesztek
    \item struktúrálható tesztek (teszt program $\to$ teszt csomag $\to$ teszteset)
    \item informatív
    \item leveszi a tesztelés technikai részének terhét a tesztelőről
    \item gyors (megosztott erőforrások)
    \item könnyen tanulható (xUnit architektúra)
  \end{itemize}
\end{frame}

\begin{frame}
  Telepítés (Ubuntu 18.04 LTS)
  \begin{description}[m]
    \item[\texttt{sudo apt install libgtest-dev}] \hfill \\ Teszt keretrendszer forrásainak beszerzése.
    \item[\texttt{sudo apt install cmake}] \hfill \\ Ezzel végezzük a forráskódok automatizált fordítását.
    \item[\texttt{cd /usr/src/gtest}] \hfill \\ Ebben a mappában találhatóak a források.
    \item[\texttt{sudo cmake CMakeLists.txt}] \hfill \\ Összeállító (build) környezet előkészítése.
    \item[\texttt{sudo make}] \hfill \\ Összeállítás indítása.
  \end{description}
\end{frame}

\begin{frame}
  \begin{description}[m]
    \item[\texttt{sudo ln -st /usr/lib/ /usr/src/gtest/libgtest.a}]
    \item[\texttt{sudo ln -st /usr/lib/ /usr/src/gtest/libgtest\_main.a}] \hfill \\ Szimbolikus hivatkozások létrehozása.
  \end{description}
  \vfill
  Feladat
  \begin{itemize}
    \item[] Készítsünk mátrixműveleteket megvalósító osztályt, ami elsőként egy mátrixszorzást valósít meg.
  \end{itemize}
\end{frame}

\begin{frame}
  \begin{exampleblock}{\textattachfile{01/matrix01.h}{matrix01.h}}
    \footnotesize
    \lstinputlisting[style=cpp,linerange={1-13},numbers=left,firstnumber=1]{01/matrix01.h}
  \end{exampleblock}
\end{frame}

\begin{frame}
  \begin{exampleblock}{\textattachfile{01/matrix01.h}{matrix01.h}}
    \lstinputlisting[style=cpp,linerange={14-19},numbers=left,firstnumber=14]{01/matrix01.h}
  \end{exampleblock}
\end{frame}

\begin{frame}
  \begin{exampleblock}{\textattachfile{01/matrix01.h}{matrix01.h}}
    \lstinputlisting[style=cpp,linerange={21-29},numbers=left,firstnumber=21]{01/matrix01.h}
  \end{exampleblock}
\end{frame}

\begin{frame}
  \begin{exampleblock}{\textattachfile{01/matrix01.h}{matrix01.h}}
    \footnotesize
    \lstinputlisting[style=cpp,linerange={31-43},numbers=left,firstnumber=31]{01/matrix01.h}
  \end{exampleblock}
\end{frame}

\begin{frame}
  \begin{exampleblock}{\textattachfile{01/matrix01.h}{matrix01.h}}
    \small
    \lstinputlisting[style=cpp,linerange={45-56},numbers=left,firstnumber=45]{01/matrix01.h}
  \end{exampleblock}
\end{frame}

\begin{frame}
  \begin{exampleblock}{\textattachfile{01/example01.cpp}{example01.cpp}}
    \small
    \lstinputlisting[style=cpp,linerange={1-13},numbers=left,firstnumber=1]{01/example01.cpp}
  \end{exampleblock}
\end{frame}

\begin{frame}[fragile]
  \begin{exampleblock}{\textattachfile{01/example01.cpp}{example01.cpp}}
    \small
    \lstinputlisting[style=cpp,linerange={14-21},numbers=left,firstnumber=14]{01/example01.cpp}
  \end{exampleblock}
  \begin{block}{Kimenet}
    \begin{verbatim}
50      50      50
90      90      90
130     130     130
\end{verbatim}
  \end{block}
\end{frame}

\begin{frame}
  Készítsünk az \texttt{example01.cpp} alapján googletest alapú tesztprogramot!
  \begin{exampleblock}{\textattachfile{01/matrix01test.cpp}{matrix01test.cpp}}
    \footnotesize
    \lstinputlisting[style=cpp,linerange={1-12},numbers=left,firstnumber=1]{01/matrix01test.cpp}
  \end{exampleblock}
\end{frame}

\begin{frame}
  \begin{exampleblock}{\textattachfile{01/matrix01test.cpp}{matrix01test.cpp}}
    \lstinputlisting[style=cpp,linerange={13-20},numbers=left,firstnumber=13]{01/matrix01test.cpp}
  \end{exampleblock}
\end{frame}

\begin{frame}
  \begin{exampleblock}{\textattachfile{01/matrix01test.cpp}{matrix01test.cpp}}
    \footnotesize
    \lstinputlisting[style=cpp,linerange={21-33},numbers=left,firstnumber=21]{01/matrix01test.cpp}
  \end{exampleblock}
\end{frame}

\begin{frame}
  \begin{exampleblock}{\textattachfile{01/CMakeLists.txt}{CMakeLists.txt}}
    \footnotesize
    \lstinputlisting[style=cpp,linerange={1-10},numbers=left,firstnumber=1]{01/CMakeLists.txt}
  \end{exampleblock}
\end{frame}

\begin{frame}[fragile]
  \begin{description}[m]
    \footnotesize
    \item[\texttt{cmake CMakeLists.txt}] \hfill \\ Összeállító (build) környezet beállítása.
    \item[\texttt{make}] \hfill \\ Összeállítás indítása.
    \item[\texttt{./runTests}] \hfill \\ Tesztprogram indítása.
  \end{description}
  \begin{block}{Kimenet}
    \scriptsize
    \begin{verbatim}
[==========] Running 1 test from 1 test case.
[----------] Global test environment set-up.
[----------] 1 test from MulTest
[ RUN      ] MulTest.meaningful
[       OK ] MulTest.meaningful (0 ms)
[----------] 1 test from MulTest (0 ms total)

[----------] Global test environment tear-down
[==========] 1 test from 1 test case ran. (0 ms total)
[  PASSED  ] 1 test.
\end{verbatim}
  \end{block}
\end{frame}

\begin{frame}
  \begin{description}[m]
    \item[Teszteset (test case)] \hfill \\ "A set of preconditions, inputs, actions (where applicable), %
      expected results and postconditions, developed based on test conditions."\\
      (meaningful, ld. \texttt{matrix01test.ccp} 5. sor)
    \item[Tesztkészlet (test suite)] \hfill \\ "A set of test cases or test procedures to be executed in a specific test cycle."\\
      (MulTest, ld. \texttt{matrix01test.ccp} 5. sor)
    \item[Tesztprogram (test program)] \hfill \\ Egy vagy több tesztkészletet foglal magába.
  \end{description}
\end{frame}

\begin{frame}
  \begin{description}[m]
    \item[Assertion ($\approx$ állítás, követelés)] Ellenőrizzük valamely elvárásunk teljesülését $\to$ siker (success), %
      nem végzetes hiba (nonfatal failure), végzetes hiba (fatal failure).\\
      Makrók:
      \begin{description}[m]
        \item[\texttt{EXPECT\_*}] nem végzetes hibát generál, ajánlott (több hiba jelezhető egyszerre)
        \item[\texttt{ASSERT\_*}] végzetes hibát generál, azonnal leállítja a tesztesetet (nincs értelme a folytatásnak; pl. ha két mátrix nem azonos méretű, nincs értelme az elemeiket összehasonlítgatni). \kiemel{Erőforrások felszabadítása, takarítás is elmarad!}
      \end{description}
  \end{description}
\end{frame}

\begin{frame}
  Rontsuk el a kódot! (,,Elfelejtjük'' összegezni a szorzatokat.)
  \begin{exampleblock}{\textattachfile{02/matrix02.h}{matrix02.h} %
    (\textattachfile{02/matrix02test.cpp}{matrix02test.cpp}, %
    \textattachfile{02/CMakeLists.txt}{CMakeLists.txt})}
    \lstinputlisting[style=cpp,linerange={45-51},numbers=left,firstnumber=45]{02/matrix02.h}
  \end{exampleblock}
\end{frame}

\begin{frame}[fragile]
  \begin{block}{Kimenet}
    \footnotesize
    \begin{verbatim}
[==========] Running 1 test from 1 test case.
[----------] Global test environment set-up.
[----------] 1 test from MulTest
[ RUN      ] MulTest.meaningful
/home/wajzy/Dokumentumok/gknb_intm006/GKxB_INTM006/02/matrix02test.cpp:25: Failure
      Expected: expected[row][col]
      Which is: 50
To be equal to: multiplied.get(row, col)
      Which is: 0
...
\end{verbatim}
  \end{block}
\end{frame}

\begin{frame}[fragile]
  \footnotesize
  \begin{block}{Kimenet}
    \begin{verbatim}
...
/home/wajzy/Dokumentumok/gknb_intm006/GKxB_INTM006/02/matrix02test.cpp:25: Failure
      Expected: expected[row][col]
      Which is: 130
To be equal to: multiplied.get(row, col)
      Which is: 0
[  FAILED  ] MulTest.meaningful (1 ms)
[----------] 1 test from MulTest (1 ms total)

[----------] Global test environment tear-down
[==========] 1 test from 1 test case ran. (1 ms total)
[  PASSED  ] 0 tests.
[  FAILED  ] 1 test, listed below:
[  FAILED  ] MulTest.meaningful

 1 FAILED TEST
\end{verbatim}
  \end{block}
\end{frame}

\begin{frame}
  Most rontsuk el másképp a kódot! (Túl nagy lesz az eredmény mátrix.)
  \begin{exampleblock}{\textattachfile{03/matrix03.h}{matrix03.h} %
    (\textattachfile{03/CMakeLists.txt}{CMakeLists.txt})}
    \lstinputlisting[style=cpp,linerange={40-44},numbers=left,firstnumber=40]{03/matrix03.h}
  \end{exampleblock}
\end{frame}

\begin{frame}
  \begin{exampleblock}{\textattachfile{03/matrix03test.cpp}{matrix03test.cpp}}
    \footnotesize
    \lstinputlisting[style=cpp,linerange={21-33},numbers=left,firstnumber=21]{03/matrix03test.cpp}
  \end{exampleblock}
\end{frame}

\begin{frame}[fragile]
  \begin{block}{Kimenet}
    \tiny
    \begin{verbatim}
[==========] Running 1 test from 1 test case.
[----------] Global test environment set-up.
[----------] 1 test from MulTest
[ RUN      ] MulTest.meaningful
/home/wajzy/Dokumentumok/gknb_intm006/GKxB_INTM006/03/matrix03test.cpp:21: Failure
      Expected: expected.size()
      Which is: 3
To be equal to: multiplied.getRowCount()
      Which is: 6
A sorok szama elter! Elvart: 3, kapott: 6
[  FAILED  ] MulTest.meaningful (0 ms)
[----------] 1 test from MulTest (0 ms total)

[----------] Global test environment tear-down
[==========] 1 test from 1 test case ran. (0 ms total)
[  PASSED  ] 0 tests.
[  FAILED  ] 1 test, listed below:
[  FAILED  ] MulTest.meaningful

 1 FAILED TEST
\end{verbatim}
  \end{block}
  \begin{itemize}
    \item Az \texttt{ASSERT\_EQ} leállította a tesztesetet.
    \item Testreszabott hibaüzeneteket jelenítettünk meg.
  \end{itemize}
\end{frame}

\section{Források}

\begin{frame}
  Tesztelésről általában\\
  \hiv{\href{https://www.tankonyvtar.hu/hu/tartalom/tamop425/0046\_szoftverteszteles/index.html}%
  {Ficsor Lajos, Kovács László, Kusper Gábor, Krizsán Zoltán: Szoftvertesztelés}}\\
  \hiv{\href{https://hstqb.org/downloadarea/istqb-ctfl-syllabus-2018-magyar/}{ISTQB CTFL Syllabus 2018}}\\
  \hiv{\href{https://glossary.istqb.org/en/search/}{Szakkifejezések kereshető gyűjteménye}}\\
  \vfill
  googletest\\
  \hiv{\href{https://github.com/google/googletest/blob/master/googletest/docs/primer.md}%
    {Hivatalos Google tutorial, bevezető}}\\
  \hiv{\href{https://github.com/google/googletest/blob/master/googletest/docs/advanced.md}%
    {Hivatalos Google tutorial, fejlett technikák}}\\
  \hiv{\href{https://www.eriksmistad.no/getting-started-with-google-test-on-ubuntu/}%
    {Ubuntu-specifikus részletek}}\\
  \hiv{\href{https://developer.ibm.com/articles/au-googletestingframework/}{IBM tananyag a googletest-hez}}\\
\end{frame}

\end{document}
